\documentclass{article}
\usepackage[utf8]{inputenc}
\usepackage{graphicx}
\usepackage{fancyhdr}
\usepackage[margin=3cm,bottom=3cm,top=3cm]{geometry}
\usepackage[table]{xcolor}
\usepackage{tabularx}

\pagestyle{fancy}
\fancyhf{}
\fancyhead[L]{The CheRocky Project}
\fancyhead[R]{Verbale Interno 22/11/19}
\fancyfoot[R]{\thepage}
% Colore rosso custom
\definecolor{custom}{RGB}{192,53,53}
%Header
\renewcommand{\headrulewidth}{4pt}
\renewcommand{\headrule}{\hbox to\headwidth{\color{custom}\leaders\hrule height \headrulewidth\hfill}}
% Footer
\renewcommand{\footrulewidth}{1pt}
\renewcommand{\footrule}{\hbox to\headwidth{\color{custom}\leaders\hrule height \footrulewidth\hfill}}

\renewcommand{\labelenumii}{\theenumii} % Per avere i menu innestati con i numeri
\renewcommand{\theenumii}{\theenumi.\arabic{enumii}.} % Per avere i menu innestati con i numeri

% Colore righe tabelle
\arrayrulecolor{custom}

\begin{document}

    % Frontespizio
    \begin{titlepage}
        \centering
        \vspace*{\fill}
        \begin{figure}[http]
            \centering
            \includegraphics[width=10cm]{logoGruppo.jpg}
        \end{figure}
        
        \vspace*{0.5cm}
    
        \huge\bfseries\centerline{Verbale Interno 22/11/19}
        \small{cherockyproj@gmail.com}
        \vspace*{1cm}
        \normalsize\mdseries{
            \begin{center}
                \begin{tabular}{ l || l }
                 \bfseries Versione & 0.0.1 \\ 
                 \bfseries Approvazione & n.d. \\  
                 \bfseries Redazione & Mirco Giardina \\
                 \bfseries Verifica & n.d. \\
                 \bfseries Stato & Verificare \\
                 \bfseries Uso & Interno \\
                 \bfseries Destinato a & The CheRocky Project\\
                 & Professore Tullio Vardanega \\
                 & Professore Riccardo Cardin
                \end{tabular}
            \end{center}
        }
        
        \vspace*{\fill}
        {\bfseries Descrizione:}
        Riassunto dell’incontro del gruppo The CheRocky Project tenutosi il 22/11/2019.
    \end{titlepage}
    
    % Pagina cronologia modifiche
    \vspace*{0.3cm}
    {\Large\bfseries Cronologia delle modifiche}
    \newline\newline
    % Impostazioni tabella
    \setlength{\tabcolsep}{18pt}
    \renewcommand{\arraystretch}{1.5}
    {\rowcolors{2}{black!5}{black!20} \normalsize\mdseries
        \begin{tabular}{ |p{1,2cm} p{1cm} p{2,5cm} p{2cm} p{3cm}| } 
            \hline
            \rowcolor{custom}
            \bfseries\leavevmode\color{white} Versione & \bfseries\leavevmode\color{white} Data & \bfseries\leavevmode\color{white} Nominativo & \bfseries\leavevmode\color{white} Ruolo & \bfseries\leavevmode\color{white} Descrizione \\
            \hline\hline
            0.0.1 & 23/11/19 & Mirco Giardina & Analista & Creato documento \\ 
            \hline
        \end{tabular}
    }
    
    % Pagina indici
    \clearpage
    \renewcommand*\contentsname{Indice}
    \tableofcontents
    
    % Contenuto
    \newpage
    \section{Informazioni generali}
    \subsection{Informazioni incontro}
    \begin{itemize}
        {\item\bfseries Luogo:}  Aula 1C150 Torre Archimede
        {\item\bfseries Data:} 22/11/2019
        {\item\bfseries Ora di inizio:} 14:30
        {\item\bfseries Ora di fine:} 17:30
        {\item\bfseries Partecipanti:}
            \begin{itemize}
                \item Alessio Bettarella
                \item Francesco Freda
                \item Mirco Giardina
                \item Mattia Gottardello
                \item Daniele Larosa
                \item Nicola Costa
                \item Nicola Panozzo
                \item Ilaria Peron
            \end{itemize}
    \end{itemize}
    \subsection{Argomenti Trattati}
    Lo scopo del primo incontro è stato quello di conoscersi, e di esporre opinioni ed interessi riguardanti i capitolati presentati.\'E stata stilata una lista dei nomi possibili del gruppo e scelto il più gettonato mediante una votazione anonima. Successivamente ogni membro del gruppo ha discusso pro e contro riguardo i capitolati di maggior interesse. Di comune accordo i membri del gruppo hanno deciso di partecipare a tutti i seminari tenuti dalle aziende proponenti in modo da raccogliere maggiori informazioni per poter fare una scelta consapevole. In questa fase iniziale è emerso un interesse maggiore per i capitolati C1 (Autonomous Highlights Platform) e C6 (ThiReMa - Things Relationship Management). Sono poi stati disccussi i possibili strumenti da utilizzare basandosi su ricerce online e sull'esperienza di alcuni membri del gruppo ed è stato concluso che utilizzieremo:
	\begin{itemize}
	    {\item\bfseries Git:} per il versionamento, inoltre consente di vedere i diff tra due versioni così da permettere al verificatore di leggere solo le correzioni apportate
	    {\item\bfseries GitHub:} come interfaccia web e server di hosting per git; permette l'assegnazione di permessi e la gestione dei compiti mediante issues e boards.
	    {\item\bfseries GitKraken:} come software per il computer utilizzabile senza dover passare per l'interfaccia web di github. Compatibile con i principali sistemi operativi.
	    {\item\bfseries Google Drive:} per la condivisione di materiale all'interno del gruppo
	    {\item\bfseries Slack:} per la comunicazione tra i membri del gruppo mediante canali suddivisi per argomento
	\end{itemize}
	Sono state inoltre fissate le date per le future riunioni in base agli impregni di tutti.
	L'incontro è terminato incaricando i membri del team ad informarsi riguardo a plugin da poter utilizzare su github.
	\newpage
	\section{Verbale della riunione}
	\begin{itemize}
	    \item Primo incontro conoscitivo
	    \item Discussione sui capitolati proposti: pro, contro ed interessi individuali
	    \item Discussione e scelta di nome, logo ed e-mail del gruppo
	    \item Schedulazione delle future riunioni del gruppo
	    \item Discussione, valutazione e scelta di uno strumento per la comunicazione dei membri del gruppo
        \item Discussione, valutazione e scelta di uno strumento per la memorizzazione e condivisione dei documenti
        \item Discussione, valutazione e scelta del software di controllo versione distribuito
	\end{itemize}
    \section{Riepilogo decisioni}
    \setlength{\tabcolsep}{18pt}
    \renewcommand{\arraystretch}{1.5}
    {\rowcolors{2}{black!5}{black!20} \normalsize\mdseries
    \begin{tabular}{ |p{1,5cm} p{11,5cm}| } 
        \hline
        \rowcolor{custom}
        \bfseries\leavevmode\color{white} Codice & \bfseries\leavevmode\color{white} Decisione \\
        \hline\hline
        VI\_1.1 & Scelto The Cherocky Project come nome del gruppo \\
        VI\_1.2 & Scelto C1 come capitolato principale \\
        VI\_1.3 & Scelto git come strumento di versionamento \\
        VI\_1.4 & Scelto GitHub come servizio di host per git \\
        VI\_1.5 & Scelto GitKraken come interfaccia software per git \\
        VI\_1.6 & Scelto GoogleDrive come strumento di condivisione dei file \\
        VI\_1.7 & Scelto Slack come strumento per la comunicazione all'interno del gruppo \\
        VI\_1.8 & Scelti i giorni in cui trovarsi nelle prossime settimane \\
        \hline
    \end{tabular}
    }
\end{document}
